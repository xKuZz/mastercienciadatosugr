Profesor: Julián Luengo 
Tutorías: Edificio clases Despacho 1.11 L 17:30-19:30 X 10:00-14:00

Modelo 1 : knn para clasificación
Medida de distancia: No siempre es lo ideal utilizar la distancia euclídea.
¿Qué hacemos si tenemos un predictor categórico?
BAJO MEDIO ALTO (hay orden) -> 1 2  3
Atributos sin orden -> crear dummys (una columna, atributo binario, por categoría) [ej color: rojo verde azul]
Otra manera sería utilizar medidas de distancia que utilizan atributos nominales. Ej: distancia HVDM

Otro problema de distancia euclídea o de Minkowski:
Al añadir muchos atributos empieza a dar la misma distancia con todas las muestras?? (más de 15-16) ? (La maldición de la dimensionalidad).
Diferencia de escalas en variable afecta al modelo, normalizar antes (transformación min-max o normalización de media 0 y desviación 1.
Podemos darle peso o quitar atributos (lo veremos en el curso de preprocesamiento).

¿Qué pasa con el valor de k?
Mapa de Voronoi (para 2 dimensiones, k=1) <- Para cada punto representa todos los puntos del espacio que son los más cercanos a él de un color. Es el único caso en el que podemos considerar "interpretable" el modelo.
Problema de k = 1, posibilidad de solapamientos.

k más bajo <- más preciso en training | k más alto <- más generalizado en training
aumentar k para evitar overfitting en knn
Tenemos que probar nosotros cuál es la k ideal para nuestro problema.

Problema abierto para el futuro (knn): Podemos sólo quedarnos con los ejemplos necesarios para hacer una clasificación correcta.

Problemas: Alto coste computacional para cálculo de distancia, alto consumo de memoria.
Normalmente k-nn debe ir asociado a pocos ejemplos, pocos atributos.


EVALUACIÓN
------------------
Aplicar los métodos que hemos visto (k-nn, lda y qda) [no regresión logística]
Prueba con algunos k distintos, probar medidas de distancia distinta. Contar bien la historia y los paquetes usados.


REGRESIÓN LOGÍSTICA
-------------------
Si aplicamos un regresor lineal podemos encontrarnos con probabilidades negativas o probabilidades superiores a 1. Utilizar una regresión lineal no es una buena idea por esos valores extremos.

Para arreglarlo => Si la regresión lineal es 
$$\beta_0 + \beta_1x$$
Lo pasamos a exponencial y dividemos entre eso más 1.
$$\frac{e^{(\beta_0 + \beta_1x)}}{1 + e^{(\beta_0 + \beta_1x)}}$$

OPTIMIZAR SIGMOIDE
$$\frac{P(Y=1|X)}{1-P(Y=1|X)}$$ Si la división vale 1 misma probabilidad más de 1 es factor veces más probable la clase 1, menor que 1 más veces probable la clase 0.

$$\frac{P(Y=1|X)}{1-P(Y=1|X)} = e^{\beta_0+\beta_1x}$$

$$ ln (z) = \beta_0+\beta_1x$$

a ln(z) se llama logit en estadística

INTERPRETACIÓN DE LOS COEFICIENTES BETAS
$\beta_1$ nos da información sobre la relación lineal entre la variable de entrada y la de salida.

family="binomial" (R, sólo tenemos dos clases), family=""
Se hace un Z test para comprobar. Cuantos más asteriscos mejor :D

Regresión logística crea dos niveles de respuesta para valores categóricos??

Si es binomial optimiza:
y * ^y + (1-y)(1-^y)

La búsqueda de los parámetros Beta es iterativa y aproximada.

LINEAR DISCRIMINANT ANALYSIS
-----------------------------

Vamos a usar el teorema de Bayes, necesitamos las distribución conjunta de valores, es muy costosa computacionalmente (test exponencial) (incluso para dos clases).

Sólo existe una solución única, inversas de matrices.
Hay qu tener más ejemplos que clases y los predictores asume que han de ser independientes.
Reescalado y normalización

Recomienda 5-10 veces más muestras que atributos.


knn, lda y qda
Explorar distinto valor de k.
Explorar si se cumplen asunciones de normalidad.
Comparar los tres, utilizar Friedman + post-hoc Holm y ver si son significativamente diferentes. Hacer con validación cruzada los valores para el test. Usar particiones preplanteadas?
